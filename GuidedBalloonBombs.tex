\documentclass[12pt]{article}
\usepackage[margin=1in]{geometry}
\usepackage{multicol}

\begin{document}
\begin{center}
\begin{LARGE}
\textbf{Proposal Abstract in Response to DARPA-BAA-16-22:} \\
\textbf{``Guided Balloon Bombs''} \\~\\~\\
\end{LARGE}
\begin{Large}
\textbf{Lead Organization: DERPA} \\~\\
\end{Large}
\end{center}

\begin{multicols}{2}
\begin{Large}
\begin{center}
\textbf{Technical POC} \\
John McFarland Lynch \\
208 Cobble Place \\
Durham, NC 27712 \\
(919) 471-8975 \\
lychrel@ghostmail.com \\~\\

\textbf{Administrative POC} \\
Ryan Warren Lynch \\
208 Cobble Place \\
Durham, NC 27712 \\
(919) 471-8975 \\
ryanlynchtds@gmail.com

\vfill\columnbreak

\textbf{Security POC} \\
Michael Arcidiacono \\
4217 Forest Edge Trail \\
Durham, NC 27705 \\
(919) 909-2915 \\
mixarcidiacono@gmail.com \\~\\~\\~\\

\textbf{CAGE Code} \\
\textbf{7LH20}

\end{center}
\end{Large}
\end{multicols}
\pagebreak

\section{Goals and Impact}

The goal of this research is to investigate methods by
which violent non-state actors might cheaply and easily make
targeted airborne IED attacks against military personnel, their
equipment, and their locations in foreign countries.

Normally, launching or dropping explosives from the air (in the
form of bombs or ballistic missiles, for example) is too expensive
and complicated to be pursued by non-state actors. This research outlines
an airborne IED that can be produced from commercially available goods
with comparatively little technical experience. Through the use of a
GPS-guided weather-balloon rig, the explosive could be carried at a
distance of up to 20 miles above-ground; it would release from the rig
and fall to its target, detonating upon impact. At 20 miles above-ground,
the bomb would evade radar detection, and the weather-balloon rig could be
configured to launch at any altitude below that.

The ability to cheaply and quickly drop IEDs from the air would enable enemy
combatants a much easier and more damaging method of attacking military
installments (bases, camps) and equipment. They would be able to inflict
explosive structural damage to buildings and equipment, in addition to
fatal bodily harm to personnel, all from a safe distance. Without any on-board
guidance systems, the bombs would be relatively quiet in their descent, and they
would also be small enough in size and density to avoid detection.

With this technology realized, even the smallest non-state actors would be able
to launch remote attacks against a wide variety of U.S. military installations
in foreign countries, threatening those installations' safety and peace of mind.
This research would investigate the feasibility of such a design.

\section{Technical Plan}

Each bomb would be carried up to 20 miles into the air by PVC frame fitted
with a cluster of weather balloons (1200g; Scientific Sales, model \#8244,
for example) filled with hydrogen (which is easier to obtain than helium).
The bomb would attach to the weather-balloon rig through the use of a worm
gear. (Our design for this stage is a highly simplified version of the mechanism
detailed in patent US7845263B1.)

The weather-balloon rig would utilize an altitude sensor to determine when
it had achieved the correct predesignated altitude; once there, it would move
toward a set of predesignated GPS coordinates through the use of four motors
(such as Turnigy RotoMax 150cc Brushless Outrunner motors) with propellers
(for example, 32x12 Aerostar Gas Series Wood Propellers) oriented in a diamond
shape around the robot. These would be controlled by a flight controller
(such as a Turnigy dlux 250A HV 14s 60v ESC).

Once at the correct location, the worm gear would turn; the bomb would release
from the rig, drop to its target, and detonate with the energy supplied from
impact. Dropped from 20 miles above-ground, the bomb would reach an impact
velocity of over 700 meters per second; dropped from just 5 miles, its impact
velocity would still exceed the speed of sound.

The bomb could accommodate a variety of different IED designs. For example, 200
pounds of Tannerite (which can be purchased at a cost of \$425 per 250 pounds from
Ammonium Nitrate Company, or synthesized independently from ammonium nitrate and
aluminum powder), would produce a detonation velocity of around 2,980 m/s and
roughly 0.5 gigajoules of energy upon impact. The rig's delivery system would
deliver enough impact force to detonate many unstable substances, so it could
utilize a variety of different IED designs.

The hydrogen for the weather balloons could be synthesized cheaply and quickly
from water with the use of an electrolysis machine that could be assembled
entirely from Home Depot parts and materials and which could produce hydrogen
gas at a rate of 5 liters per minute with two car batteries used for power.
If you would like specifics on this design, please email our Technical Point of Contact.

The balloon rig would likely use an Arduino Uno or similar microcontroller,
and would contain a GPS, an IMU, and an altitude sensor. The  Arduino (or
similar microcontroller) would determine when to release the rocket and where
to move the balloon rig. If necessary, a smartphone could substitute for the
microcontroller and IMU.

If necessary or desired, the PVC rig could be constructed with an extra linear
actuator (a rod to which the balloons are affixed) such that, when the bomb is
dropped, the actuator would move and release most of the balloons. This would
enable retrieval of the rig: instead of escaping into the upper atmosphere, it
could drive towards a predesignated set of ``home'' coordinates as it lost altitude.

Feasibility studies could be conducted without the use of actual explosives for
the bomb, so no state explosives laws would be infringed.


\section{Capabilities/Management Plan}

DERPA consists largely of rising college freshmen with experience in the design
and creation of robotic systems and computer programming. All members partake in
the group's creative process, and members contribute individually to the areas of
each project with which they are most familiar. DERPA's small size and loose
administrative structure enable fluid cooperation and quick communication among
members, which enhances our ability to work through designs thoroughly, quickly,
and with multiple perspectives.

For this project, we anticipate Michael Arcidiacono, Jonathan Stepp, and John
Lynch to work the most on the missile and balloon-rig's programming, with Ryan Lynch,
Matthew Miller, and David Proa\~{n}o heading the creation of its chassis and missile.
However, these distinctions are largely arbitrary; every member will involve himself
extensively in the whole process.

\end{document}
